\usepackage{tikz}
\usepackage{pgfplots}
\pgfplotsset{compat=1.12}

% you can add draft to documentclass to avoid recompiling tikz/pgfplots
\usetikzlibrary{external} 
\tikzexternalize[prefix=precompiled/]

%\usetikzlibrary{external} \tikzset{external/system call={pdflatex
     %\tikzexternalcheckshellescape -halt-on-error
     %-interaction=batchmode -jobname "\image" "\texsource"}}
 %\tikzexternalize[prefix=precompiled/]

% CHANGE pdflatex command from 
%		pdflatex.exe
%		-synctex=-1 -interaction=nonstopmode "%pm"
% to
%		pdflatex.exe
%   -synctex=-1 -interaction=nonstopmode -shell-escape "%pm"
%
% Source: http://tex.stackexchange.com/questions/170291/tikz-wont-let-me-externalize-and-halts-on-error
%\colorlet{mycolor1}{blue!60!black}
%\colorlet{mycolor2}{red!60!black}

% matlab colors
\definecolor{mycolor1}{rgb}{0, 0.4470, 0.7410} % 0, 114, 189
\definecolor{mycolor2}{rgb}{0.8500, 0.3250, 0.0980}
\definecolor{mycolor3}{rgb}{0.9290, 0.6940, 0.1250}
\definecolor{mycolor4}{rgb}{0.4940, 0.1840, 0.5560}
\definecolor{mycolor5}{rgb}{0.4660, 0.6740, 0.1880}
\definecolor{mycolor6}{rgb}{0.3010, 0.7450, 0.9330}
\definecolor{mycolor7}{rgb}{0.6350, 0.0780, 0.1840}

\colorlet{mycolor1Light}{mycolor1!15}
\colorlet{mycolor2Light}{mycolor2!15}

% include FSMs
\usetikzlibrary{arrows, automata, bending, calc, fit, positioning, shapes, backgrounds}

% include dsp
\makeatletter
\input{tikz_pkgs/dsp/tikzlibrarydsp.code.tex}
\makeatother

\usetikzlibrary{chains}

%globally plotting settings
\pgfplotsset{/pgf/number format/use comma}

\tikzset{%
	% move x and y axis label
	every x tick label/.style={yshift=-0.4em},
	every y tick label/.style={xshift=-0.1em}
}

%%%%%%%%%%%%%%%%%%%%%%%%%%%%%%%%%%%%%%%%%%%%%%%%%%%%%%%%%%
% for flowcharts
% see http://www.texample.net/tikz/examples/simple-flow-chart/

% Define block styles
\tikzstyle{flowdecision} = [diamond, draw, 
    text width=6em, text badly centered, inner sep=0pt, aspect=2]
\tikzstyle{flowblock} = [rectangle, draw, 
    text width=8em, text centered, rounded corners,
		]
\tikzstyle{flowline} = [draw, -latex']
\tikzstyle{flowconn} = [draw]
\tikzstyle{flowcloud} = [draw, ellipse,
	node distance=3cm,
    minimum height=2em]

%%%%%%%%%%%%%%%%%%%%%%%%%%%%%%%%%%%%%%%%%%%%%%%%%%%%%%%%%%

%% align two tikz picture side by side
% syntax: 
%	\twotikzPictures{<figure location specifier>}
%		{<figure1>}{<caption1>}{<label1>}
%		{<figure2>}{<caption2>}{<label2>}{<caption>}{<label>}
\newcommand{\twotikzPictures}[9]{
  \begin{figure}[#1]
    \begin{minipage}[t]{.5\linewidth}
      \centering #2%
    \end{minipage}%
    \begin{minipage}[t]{.5\linewidth}
      \centering #5%
    \end{minipage}
    \begin{minipage}[t]{.5\linewidth}
      \subcaption{#3}
      \label{#4}
    \end{minipage}%
    \begin{minipage}[t]{.5\linewidth}
      \subcaption{#6}
      \label{#7}
    \end{minipage}
		\caption{#8}
		\label{#9}
  \end{figure}
}

%%%%%%%%%%%%%%%%%%%%%%%%%%%%%%%%%%%%%%%%%%%%%%%%%%%%%%%%%%
% use this size for matlab2tikz

\newcommand{\tikzpictureheightONETIKZPICTURE}{4.2cm}

\newcommand{\tikzpicturewidthTWOTIKZPICTURES}{0.7\linewidth}
\newcommand{\tikzpictureheightTWOTIKZPICTURES}{\tikzpictureheightONETIKZPICTURE}



%%%%%%%%%%%%%%%%%%%%%%%%%%%%%%%%%%%%%%%%%%%%%%%%%%%%%%%%%%

%% little helper macro for tikzpictures

\newcommand{\tikzaxislabel}[2]{$#1\textrm{ / }\si{#2}$}

%%%%%%%%%%%%%%%%%%%%%%%%%%%%%%%%%%%%%%%%%%%%%%%%%%%%%%%%%%

%% style to highlight pgfplots
% http://tex.stackexchange.com/questions/103302/pgfplots-highlighting-filling-an-area-in-a-data-file
%
% does not work with legends!
\pgfkeys{%
  /tikz/on layer/.code={
    \pgfonlayer{#1}\begingroup
    \aftergroup\endpgfonlayer
    \aftergroup\endgroup
  }
}

\pgfplotsset{
    highlight/.code args={#1:#2}{
        \fill [every highlight] ({axis cs:#1,0}|-{rel axis cs:0,0}) rectangle ({axis cs:#2,0}|-{rel axis cs:0,1});
    },
    /tikz/every highlight/.style={
        on layer=\pgfkeysvalueof{/pgfplots/highlight layer},
        mycolor1Light
    },
    /tikz/highlight style/.style={
        /tikz/every highlight/.append style=#1
    },
    highlight layer/.initial=axis background
}


