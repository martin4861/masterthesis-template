\usepackage{tikz}
\usepackage{pgfplots}
\pgfplotsset{compat=1.12}

% you can add draft to documentclass to avoid recompiling tikz/pgfplots
\usetikzlibrary{external} 
\tikzexternalize[prefix=precompiled/]

%\usetikzlibrary{external} \tikzset{external/system call={pdflatex
     %\tikzexternalcheckshellescape -halt-on-error
     %-interaction=batchmode -jobname "\image" "\texsource"}}
 %\tikzexternalize[prefix=precompiled/]

% CHANGE pdflatex command from 
%		pdflatex.exe
%		-synctex=-1 -interaction=nonstopmode "%pm"
% to
%		pdflatex.exe
%   -synctex=-1 -interaction=nonstopmode -shell-escape "%pm"
%
% Source: http://tex.stackexchange.com/questions/170291/tikz-wont-let-me-externalize-and-halts-on-error
%\colorlet{mycolor1}{blue!60!black}
%\colorlet{mycolor2}{red!60!black}

% matlab colors
\definecolor{mycolor1}{rgb}{0, 0.4470, 0.7410} % 0, 114, 189
\definecolor{mycolor2}{rgb}{0.8500, 0.3250, 0.0980}
\definecolor{mycolor3}{rgb}{0.9290, 0.6940, 0.1250}
\definecolor{mycolor4}{rgb}{0.4940, 0.1840, 0.5560}
\definecolor{mycolor5}{rgb}{0.4660, 0.6740, 0.1880}
\definecolor{mycolor6}{rgb}{0.3010, 0.7450, 0.9330}
\definecolor{mycolor7}{rgb}{0.6350, 0.0780, 0.1840}

\colorlet{mycolor1Light}{mycolor1!15}
\colorlet{mycolor2Light}{mycolor2!15}

% include FSMs
\usetikzlibrary{arrows, automata, bending, calc, fit, positioning, shapes, backgrounds}

% include dsp
\makeatletter
% -------------------------------------------------------------------------
% This library for block diagrams and signal flow graphs was inspired by
% the library "signalflow" of Dr. Karlheinz Ochs, Ruhr-University of Bochum,
% Germany. Furthermore, some ideas were taken from the library "circuitikz"
% of Massimo A. Redaelli and from the PGF library itself.
%
% Copyright 2012 by Matthias Hotz
%
% This work is licensed under the Creative Commons Attribution 2.5 Generic
% License. To view a copy of this license, visit
%          http://creativecommons.org/licenses/by/2.5/
% or send a letter to Creative Commons, 444 Castro Street, Suite 900,
% Mountain View, California, 94041, USA.
% -------------------------------------------------------------------------

\usetikzlibrary{arrows, calc, positioning, decorations.markings}

% -------------------------------------------------------------------------
% Parameters for the library

\newcommand{\dsplinewidth}{0.25mm}           % Line width for connections
\newcommand{\dspblocklinewidth}{0.3mm}       % Line width for blocks
\newcommand{\dspoperatordiameter}{4mm}       % Diameter for adder, multiplier, mixer
\newcommand{\dspoperatorlabelspacing}{2mm}   % Distance from symbol to label for adder, multiplier, mixer
\newcommand{\dspnoderadius}{1mm}             % Filled and empty node
\newcommand{\dspsquareblocksize}{8mm}        % Size for square blocks, e.g. for delay elements, decimator, expander
\newcommand{\dspfilterwidth}{14mm}           % Width of a filter block

% -------------------------------------------------------------------------
% Define new arrow heads

\pgfarrowsdeclare{dsparrow}{dsparrow}
{
	\arrowsize=0.25pt
	\advance\arrowsize by .5\pgflinewidth
	\pgfarrowsleftextend{-4\arrowsize}
	\pgfarrowsrightextend{4\arrowsize}
}
{
	\arrowsize=0.25pt
	\advance\arrowsize by .5\pgflinewidth
	\pgfsetdash{}{0pt} % Solid line (do not dash)
	\pgfsetmiterjoin	 % Fixed miter join of line
	\pgfsetbuttcap		 % Fixed butt cap of line
	\pgfpathmoveto{\pgfpoint{-4\arrowsize}{2.5\arrowsize}}
	\pgfpathlineto{\pgfpoint{4\arrowsize}{0pt}}
	\pgfpathlineto{\pgfpoint{-4\arrowsize}{-2.5\arrowsize}}
	\pgfpathclose
	\pgfusepathqfill
}

\pgfarrowsdeclare{dsparrowmid}{dsparrowmid}
{
	\arrowsize=0.25pt
	\advance\arrowsize by .5\pgflinewidth
	\pgfarrowsleftextend{-4\arrowsize}
	\pgfarrowsrightextend{4\arrowsize}
}
{
	\arrowsize=0.25pt
	\advance\arrowsize by .5\pgflinewidth
	\pgfsetdash{}{0pt}
	\pgfsetmiterjoin
	\pgfsetbuttcap
	\pgfpathmoveto{\pgfpoint{0}{2.5\arrowsize}}
	\pgfpathlineto{\pgfpoint{8\arrowsize}{0pt}}
	\pgfpathlineto{\pgfpoint{0}{-2.5\arrowsize}}
	\pgfpathclose
	\pgfusepathqfill
}

% -------------------------------------------------------------------------
% Define new node shapes

\makeatletter

\pgfkeys{/tikz/dsp/label/.initial=above}

% Generic shape generator for operators, i.e. nodes with a circular
% shape with an additional customizable drawing and a text label
\long\def\dspdeclareoperator#1#2{
	\pgfdeclareshape{#1}
	{
		% Saved anchors, macros and dimensions
		\savedanchor\centerpoint{\pgfpointorigin}
		\savedmacro\label{\def\label{\pgfkeysvalueof{/tikz/dsp/label}}}
	  \saveddimen\radius
	  {
		  \pgfmathsetlength\pgf@xa{\pgfshapeminwidth}
		  \pgfmathsetlength\pgf@ya{\pgfshapeminheight}
	    \ifdim\pgf@xa>\pgf@ya
	      \pgf@x=.5\pgf@xa
	    \else
	      \pgf@x=.5\pgf@ya
	    \fi
	  }
	  
	  % Inherit all anchors from the 'circle'-shape:
	  \inheritanchor[from={circle}]{center}
	  \inheritanchor[from={circle}]{mid}
	  \inheritanchor[from={circle}]{base}
	  \inheritanchor[from={circle}]{north}
	  \inheritanchor[from={circle}]{south}
	  \inheritanchor[from={circle}]{west}
	  \inheritanchor[from={circle}]{east}
	  \inheritanchor[from={circle}]{mid west}
	  \inheritanchor[from={circle}]{mid east}
	  \inheritanchor[from={circle}]{base west}
	  \inheritanchor[from={circle}]{base east}
	  \inheritanchor[from={circle}]{north west}
	  \inheritanchor[from={circle}]{south west}
	  \inheritanchor[from={circle}]{north east}
	  \inheritanchor[from={circle}]{south east}
	  \inheritanchorborder[from={circle}]
	  
	  % Draw circle and embed additional code
	  \backgroundpath
	  {
	    % Draw circle
	    \pgfpathcircle{\centerpoint}{\radius}
	    
	    % Embed additional code
	    % (Note that this code must call e.g. \pgfusepathqstroke
	    #2
	  }
	
		% Define anchor parametrized by the PGF key /tikz/dsp/label
	  \anchor{text}
	  {
			\centerpoint
			%
	    \def\templabelabove{above}
	    \def\templabelbelow{below}
	    \def\templabelleft{left}
	    \def\templabelright{right}
	    \pgfutil@tempdima=\dspoperatorlabelspacing
	    %
	    \ifx\label\templabelabove
				\advance\pgf@x by -0.5\wd\pgfnodeparttextbox
				\advance\pgf@y by \radius
				\advance\pgf@y by \pgfutil@tempdima
	    \fi
	    %
	    \ifx\label\templabelbelow
				\advance\pgf@x by -0.5\wd\pgfnodeparttextbox
				\advance\pgf@y by -\radius
				\advance\pgf@y by -\pgfutil@tempdima
				\advance\pgf@y by -\ht\pgfnodeparttextbox
	    \fi
	    %
	    \ifx\label\templabelleft
				\advance\pgf@x by -\radius
				\advance\pgf@x by -\pgfutil@tempdima
				\advance\pgf@x by -\wd\pgfnodeparttextbox
				\advance\pgf@y by -0.5\ht\pgfnodeparttextbox
				\advance\pgf@y by +0.5\dp\pgfnodeparttextbox
	    \fi
	    %
	    \ifx\label\templabelright
				\advance\pgf@x by \radius
				\advance\pgf@x by \pgfutil@tempdima
				\advance\pgf@y by -0.5\ht\pgfnodeparttextbox
				\advance\pgf@y by +0.5\dp\pgfnodeparttextbox
	    \fi
	  }
	}
}

\dspdeclareoperator{dspshapecircle}{\pgfusepathqstroke}

\dspdeclareoperator{dspshapecirclefull}{\pgfusepathqfillstroke}

\dspdeclareoperator{dspshapeadder}{
	% Coordinate offset for the plus
	\pgfutil@tempdima=\radius
	\pgfutil@tempdima=0.55\pgfutil@tempdima
	
	% Draw plus
	\pgfmoveto{\pgfpointadd{\centerpoint}{\pgfpoint{0pt}{-\pgfutil@tempdima}}}
	\pgflineto{\pgfpointadd{\centerpoint}{\pgfpoint{0pt}{ \pgfutil@tempdima}}}
	
	\pgfmoveto{\pgfpointadd{\centerpoint}{\pgfpoint{-\pgfutil@tempdima}{0pt}}}
	\pgflineto{\pgfpointadd{\centerpoint}{\pgfpoint{ \pgfutil@tempdima}{0pt}}}
	
	\pgfusepathqstroke
}

\dspdeclareoperator{dspshapemixer}{
	% Coordinate offset for the cross
	\pgfutil@tempdima=\radius
	\pgfutil@tempdima=0.707106781\pgfutil@tempdima
	
	% Draw cross
	\pgfmoveto{\pgfpointadd{\centerpoint}{\pgfpoint{-\pgfutil@tempdima}{-\pgfutil@tempdima}}}
	\pgflineto{\pgfpointadd{\centerpoint}{\pgfpoint{ \pgfutil@tempdima}{ \pgfutil@tempdima}}}
	
	\pgfmoveto{\pgfpointadd{\centerpoint}{\pgfpoint{-\pgfutil@tempdima}{ \pgfutil@tempdima}}}
	\pgflineto{\pgfpointadd{\centerpoint}{\pgfpoint{ \pgfutil@tempdima}{-\pgfutil@tempdima}}}
	
	\pgfusepathqstroke
}

\makeatother

% -------------------------------------------------------------------------
% Define node styles

\tikzset{dspadder/.style={shape=dspshapeadder,line cap=rect,line join=rect,
	line width=\dspblocklinewidth,minimum size=\dspoperatordiameter}}
\tikzset{dspmultiplier/.style={shape=dspshapecircle,line cap=rect,line join=rect,
	line width=\dspblocklinewidth,minimum size=\dspoperatordiameter}}
\tikzset{dspmixer/.style={shape=dspshapemixer,line cap=rect,line join=rect,
	line width=\dspblocklinewidth,minimum size=\dspoperatordiameter}}

\tikzset{dspnodeopen/.style={shape=dspshapecircle,line width=\dsplinewidth,minimum size=\dspnoderadius}}
\tikzset{dspnodefull/.style={shape=dspshapecirclefull,line width=\dsplinewidth,fill,minimum size=\dspnoderadius}}

% The fixed specification of text height and text depth is the somewhat
% unaesthetic workaround to align the text in different node at the same
% baseline. See the PGF/TikZ manual, ch. 5.1.
\tikzset{dspsquare/.style={shape=rectangle,draw,align=center,text depth=0.3em,text height=1em,inner sep=0pt,
	line cap=round,line join=round,line width=\dsplinewidth,minimum size=\dspsquareblocksize}}
\tikzset{dspfilter/.style={shape=rectangle,draw,align=center,text depth=0.3em,text height=1em,inner sep=0pt,
	line cap=round,line join=round,line width=\dsplinewidth,minimum height=\dspsquareblocksize,minimum width=\dspfilterwidth}}

% -------------------------------------------------------------------------
% Define "signal flow" lines

\tikzset{dspline/.style={line width=\dsplinewidth},line cap=round,line join=round}
\tikzset{dspconn/.style={->,>=dsparrow,line width=\dsplinewidth},line cap=round,line join=round}%line cap=rect,line join=miter}
\tikzset{dspflow/.style={line width=\dsplinewidth,line cap=round,line join=round,
  decoration={markings,mark=at position 0.5 with {\arrow{dsparrowmid}}},postaction={decorate}}}

% -------------------------------------------------------------------------
% Define various utility macros

\newcommand{\downsamplertext}[1]{\raisebox{0.1em}{$\big\downarrow$}#1}
\newcommand{\upsamplertext}[1]{\raisebox{0.1em}{$\big\uparrow$}#1}

\makeatother

\usetikzlibrary{chains}

%globally plotting settings
\pgfplotsset{/pgf/number format/use comma}

\tikzset{%
	% move x and y axis label
	every x tick label/.style={yshift=-0.4em},
	every y tick label/.style={xshift=-0.1em}
}

%%%%%%%%%%%%%%%%%%%%%%%%%%%%%%%%%%%%%%%%%%%%%%%%%%%%%%%%%%
% for flowcharts
% see http://www.texample.net/tikz/examples/simple-flow-chart/

% Define block styles
\tikzstyle{flowdecision} = [diamond, draw, 
    text width=6em, text badly centered, inner sep=0pt, aspect=2]
\tikzstyle{flowblock} = [rectangle, draw, 
    text width=8em, text centered, rounded corners,
		]
\tikzstyle{flowline} = [draw, -latex']
\tikzstyle{flowconn} = [draw]
\tikzstyle{flowcloud} = [draw, ellipse,
	node distance=3cm,
    minimum height=2em]

%%%%%%%%%%%%%%%%%%%%%%%%%%%%%%%%%%%%%%%%%%%%%%%%%%%%%%%%%%

%% align two tikz picture side by side
% syntax: 
%	\twotikzPictures{<figure location specifier>}
%		{<figure1>}{<caption1>}{<label1>}
%		{<figure2>}{<caption2>}{<label2>}{<caption>}{<label>}
\newcommand{\twotikzPictures}[9]{
  \begin{figure}[#1]
    \begin{minipage}[t]{.5\linewidth}
      \centering #2%
    \end{minipage}%
    \begin{minipage}[t]{.5\linewidth}
      \centering #5%
    \end{minipage}
    \begin{minipage}[t]{.5\linewidth}
      \subcaption{#3}
      \label{#4}
    \end{minipage}%
    \begin{minipage}[t]{.5\linewidth}
      \subcaption{#6}
      \label{#7}
    \end{minipage}
		\caption{#8}
		\label{#9}
  \end{figure}
}

%%%%%%%%%%%%%%%%%%%%%%%%%%%%%%%%%%%%%%%%%%%%%%%%%%%%%%%%%%
% use this size for matlab2tikz

\newcommand{\tikzpictureheightONETIKZPICTURE}{4.2cm}

\newcommand{\tikzpicturewidthTWOTIKZPICTURES}{0.7\linewidth}
\newcommand{\tikzpictureheightTWOTIKZPICTURES}{\tikzpictureheightONETIKZPICTURE}



%%%%%%%%%%%%%%%%%%%%%%%%%%%%%%%%%%%%%%%%%%%%%%%%%%%%%%%%%%

%% little helper macro for tikzpictures

\newcommand{\tikzaxislabel}[2]{$#1\textrm{ / }\si{#2}$}

%%%%%%%%%%%%%%%%%%%%%%%%%%%%%%%%%%%%%%%%%%%%%%%%%%%%%%%%%%

%% style to highlight pgfplots
% http://tex.stackexchange.com/questions/103302/pgfplots-highlighting-filling-an-area-in-a-data-file
%
% does not work with legends!
\pgfkeys{%
  /tikz/on layer/.code={
    \pgfonlayer{#1}\begingroup
    \aftergroup\endpgfonlayer
    \aftergroup\endgroup
  }
}

\pgfplotsset{
    highlight/.code args={#1:#2}{
        \fill [every highlight] ({axis cs:#1,0}|-{rel axis cs:0,0}) rectangle ({axis cs:#2,0}|-{rel axis cs:0,1});
    },
    /tikz/every highlight/.style={
        on layer=\pgfkeysvalueof{/pgfplots/highlight layer},
        mycolor1Light
    },
    /tikz/highlight style/.style={
        /tikz/every highlight/.append style=#1
    },
    highlight layer/.initial=axis background
}


