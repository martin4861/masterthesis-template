%% Abk�rzungen

% Vektor (damit das Aussehen leicht ver�ndert werden kann)
%\newcommand{\vek}[1]{\mathbf{#1}}

\newcommand{\vek}[1]{\vec{#1}} % standard

\newcommand{\vekex}[3]{\vec{#1}^{\,#2}_{#3}} % extended 
\newcommand{\vekdot}[3]{\dot{\vec{#1}}^{\,#2}_{#3}} % extended for derivates
\newcommand{\vekhat}[3]{\hat{\vec{#1}}^{\,#2}_{#3}} % extended for estimates

% Matrix (damit das Aussehen leicht ver�ndert werden kann)
\newcommand{\mat}[1]{\mathbf{#1}} % standard
\newcommand{\matex}[3]{\mat{#1}^{#2}_{#3}} % extended 
\newcommand{\matdot}[3]{\dot{\mat{#1}}^{#2}_{#3}} % extended for derivates
\newcommand{\mathat}[3]{\hat{\mat{#1}}^{\,#2}_{#3}} % extended for estimates

% Normalverteilung
\newcommand{\gauss}[1]{{\mathcal N} \left\{ #1 \right\} }
%\newcommand{\gauss}[1]{{\mathfrak N} \left\{ #1 \right\} }

% Erwartungswert
\newcommand{\ex}[1]{E \left[ #1 \right]}

% zwei M�glichkeiten
\newcommand{\postwo}[2]{\left\{ \begin{matrix} #1 \\ #2 \end{matrix} \right.}

% Ableitungen und Integrationen
\newcommand{\di}{\text{d}}

% Fouriertransformierte
%\newcommand{\fourier}[1]{{\mathcal F} \left\{ #1 \right\} }
\newcommand{\fourier}[1]{{\mathfrak F} \left\{ #1 \right\} }

% Dirac Delta Funktion
\newcommand{\dirac}[1]{\delta \left( #1 \right) }

% Vektor oder Matrix transponierte
\newcommand{\transposeT}{\textrm{T}} % only text
\newcommand{\transpose}{^\transposeT} % full command

\newcommand{\inversT}{-1} % only text
\newcommand{\invers}{^{\inversT}} % full command

% Kennzeichnung f�r Stichprobenmittelwert
\newcommand{\mean}[1]{\overline{#1}}

%%%%%%%%%%%%%%%%%%%%%%%%%%%%%%%%%%%%%%%%%%%%%%%%%%%%%%%%%%

\newcommand{\engltext}[1]{(englisch \textit{#1})}
\newcommand{\etal}{\textit{et al.}}

% Strich �ber Text in Textmodus
\makeatletter
\newcommand*{\textoverline}[1]{$\overline{\hbox{#1}}\m@th$}
\makeatother

%%%%%%%%%%%%%%%%%%%%%%%%%%%%%%%%%%%%%%%%%%%%%%%%%%%%%%%%%%

% Abtastschritt f�r diskrete zeitsignale
\newcommand{\symtime}{k}

% diskreter Frequenzraum Abtastschritt
\newcommand{\symfreq}{n}

% Symbol f�r eine Pose
\newcommand{\sympose}{P}
\newcommand{\symdeltapose}{\Delta P}

%%%%%%%%%%%%%%%%%%%%%%%%%%%%%%%%%%%%%%%%%%%%%%%%%%%%%%%%%%

% Angabe einer Koordinate
\newcommand{\coordXY}[2]{\textrm{X}=\SI{#1}{\meter}, \textrm{Y}=\SI{#2}{\meter}}
\newcommand{\coordXYZ}[3]{\textrm{X}=\SI{#1}{\meter}, \textrm{Y}=\SI{#2}{\meter}, \textrm{Z}=\SI{#3}{\meter}}

%%%%%%%%%%%%%%%%%%%%%%%%%%%%%%%%%%%%%%%%%%%%%%%%%%%%%%%%%%

% matrix umgebungstyp
\newenvironment{matrixenv}[0]{\begin{pmatrix}}{\end{pmatrix}}

